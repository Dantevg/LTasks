\chapter{Research}\label{research}
% This chapter, or series of chapters, delves into all technical details that are
% required to \emph{prove} your scientific hypothesis.
% It should be sufficiently detailed and precise in order for any fellow computing scientist student to be able to \emph{repeat}
% your research and therewith establish the same results / conclusions that you have obtained.
% Please note that, in order to improve readability of your thesis, you can put a part of this information also in one or
% more appendices (see Appendix \ref{appendix}).

There are many design decisions of TOP in Lua to be explored, resulting from the major differences between Clean and Lua. This research explores the design space and creates a proof-of-concept of TOP in Lua based on these decisions. The proof-of-concept is complete, when:
\begin{itemize}
    \item it features a basic implementation of tasks
    \item these tasks can be composed sequentially and in parallel (both ``and'' and ``or''), while making use of observable task values.
    \item it has a way of interaction with users (editors), and some form of user interface that is automatically generated. Minimally, the editors should be able to model tables, strings, numbers and booleans.
    % \item it has shared data sources that can be modified by these editors.
\end{itemize}

The concept of shared data sources and exceptions in TOP (\S \ref{section-top-task-value}) are out of scope for this bachelor thesis.

In the next section, we think about how to meaningfully work with the dynamic type system of Lua. After that, we look how we can represent tasks (\S \ref{section-task-values}) and task types (\S \ref{section-type-representation}). Section \ref{section-combinators} takes a look at task combinators, we discuss user interfaces in section \ref{section-editors-ui} and lastly we discuss LTasks (\S \ref{section-ltasks}).

% \newpage
\section{Task type representation}\label{section-task-types}
We need some way to represent Lua types at runtime. This typing information is used by the step combinator's type match function to decide which task it should choose.
Lua has the \lua{type} function that returns the type of the value passed as a string: \lua{type(42) == "number"}. The problem is that this does not give us detailed enough information for tables; \lua{type({10, 20})} and \lua{type({hello = "world"})} both result in just \lua{"table"}.

Tasks and editors require a more elaborate system that can distinguish types of composite values. We need to consider the way these types are written, how they are represented or stored at runtime, and how they are compared against each other. We will compare multiple ways to solve the first two considerations now, how to compare types is left for section \ref{section-combinators-type-matching}.

\subsection{LuaRocks libraries}
There are a number of libraries that come up when searching for ``types'' in LuaRocks\footnote{\url{https://luarocks.org/}}, which is the most used Lua package manager. Three of them have some way to represent composite types at runtime.
Typed\footnote{\url{https://luarocks.org/modules/SovietKitsune/typed}}, LuaStruct\footnote{\label{footnote-luastruct}\url{https://luarocks.org/modules/UlisseMini/luastruct}} and Struct.lua\footnote{\url{https://github.com/mpatraw/struct.lua}}.

\subsubsection{Typed}
Typed is a library to check a function's arguments. It gives formatted error messages containing information on what type was expected. The error messages are not interesting for this thesis, but how it represents composite types is. Arrays can be represented like the string \lua{"number[]"}, maps are written as \lua{"table<string, boolean>"}. For more complicated types, it uses schemas, for example \lua{typed.Schema('test'):field('name', 'string'):field('id', 'number')}. This represents a table that contains the string field \lua{name} and a numeric field \lua{id}.

\subsubsection{LuaStruct}
LuaStruct represents types at runtime by a default value. For example, from the LuaRocks description\footref{footnote-luastruct}:

\medskip
\begin{minted}{lua}
local person = struct {
    name = "default name",
    age  = 0
}
\end{minted}

This may be a very simple way to store composite types at runtime, but it has the obvious downside that every field must have a default value. For editors, this is not that big of a problem. But for specifying what type a task accepts, this can be very inconvenient. Furthermore, in this place the actual default value does not have any use: only its type will be used. A bigger problem for representing types of tables in this thesis is that this library is only about \textit{structs}; it does not have a way to represent arrays.

\subsubsection{struct.lua}
struct.lua works in a similar way to LuaStruct: you define a type by using a default value. As such, it suffers from the same issues as LuaStruct like not being able to represent arrays.

\subsection{Lua extensions}
\subsubsection{Typed Lua}
Maidl, Mascarenhas and Ierusalimschy \cite{maidl2014typed} designed a gradually typed extension of Lua. ``does not insert runtime checks between dynamically and statically typed parts of the program,'' but it does have its own way of representing these types in code.

\subsubsection{Pallene}
Pallene, developed by Gualandi and Ierusalimschy \cite{gualandi2020pallene}, is a ``typed companion language to the Lua scripting language.'' It is ``very close to a typed subset of Lua.'' In contrast to Typed Lua, it does sometimes keep types for runtime type checks.

\subsection{Other languages}
\subsubsection{TypeScript}
TypeScript\footnote{\url{https://www.typescriptlang.org/}} is a language that transpiles to JavaScript. Like Typed Lua, its types get removed at compile time. We can still learn from the way types are written, though.

% \newpage
\section{Tasks and task values}\label{section-task-values}

\subsection{Tasks in iTasks}
Since iTasks is written in the functional language Clean, tasks are modelled as functions. Functions in Clean are pure; they cannot mutate existing values. Instead, a task is implemented as a function that can process an event, using the information stored in the \clean{IWorld} environment, and that results in a new version of itself, the task value, any user interface changes, and the \clean{IWorld}.
In listing \ref{lst:clean_task_taskresult_types} we can see that the task function (\clean{Task}) returns a \clean{TaskResult} and an \clean{IWorld}. \clean{IWorld} carries information about the entire TOP application such as the current clock, and we will not go into it further.

\begin{figure}[ht]
    \centering
    \begin{minted}{clean}
    :: Task a (=: Task` (Event -> TaskEvalOpts -> *IWorld
            -> *(TaskResult a, *IWorld)))
    
    :: TaskResult a
        = ValueResult !(TaskValue a) !TaskEvalInfo !UIChange !(Task a)
        | ExceptionResult !TaskException
        | DestroyedResult
    
    :: TaskValue a
        = NoValue
        | Value !a !Stability
    \end{minted}
    \caption{The types \clean{Task} and \clean{TaskResult} in Clean. \clean{TaskValue} from listing \ref{lst:clean_task_value} is repeated here for convenience.}
    \label{lst:clean_task_taskresult_types}
\end{figure}

\subsection{Tasks in Lua}
Lua does not have algebraic data types. Moreover, in contrast to Clean, mutation is normal and we can keep state by using tables. We can also use coroutines which makes modelling changing tasks more convenient, as execution can halt in the middle of a function and continue later on. There are three choices to be made here: whether to model the task functionality as a coroutine or as a function, whether to store that coroutine/function in a table or leave it bare, and whether to separate the actual task value from its stability. All three choices affect each other; only a few combinations actually make sense.

\subsubsection{Functions or coroutines}
Clean has no coroutines. The way that a single task can keep state and handle multiple events during the runtime of the program is by returning a new function to handle the next event. In Lua we can keep handling events within a single coroutine. We can keep state using local variables within the coroutine. If we choose to use functions, we return the task value and stability. When using coroutines, we yield. We will use coroutines for this thesis because they are much easier to work with for modelling tasks.

\subsubsection{Tasks as tables}
Close to how iTasks works in Clean, we can model tasks as bare functions or coroutines, where the task value is returned or yielded. Making use of what Lua gives us, we can store that coroutine/function in a table alongside the task value and stability. The effect of this is that all tasks that have a reference to the task can read its value at any time. In iTasks this is limited to tasks that are linked together by a combinator. Another possibility that this opens up is that we can now define other functions that operate on this task's internals, however that goes against the principle that tasks should be autonomous.

The downside of this is that task values can now be altered from outside. TOP means that tasks are autonomous: only the task itself can set its task value, and one task should not be able to modify the value of another task. This can be solved by not exposing the task value itself, but rather a function that reads from a private task value. There are multiple ways to do this, listing \ref{lst:lua_private} shows two of them. They both make use of a closure to hide the variable. Barring use of the \lua{debug} library\footnote{The \lua{debug} library violates multiple core assumptions about Lua code \cite{luareferencemanual}, so including it in considerations would not be appropriate.}, method \ref{lst:lua_private_a} makes the \lua{count} variable truly invisible and immutable from the outside. Method \ref{lst:lua_private_b} allows us to refer to the value itself instead of having to call a getter function which makes it transparent, but its downside is that it only hides the \lua{count} variable behind a metatable. The example shows that it is possible to modify the variable with a detour.

\begin{figure}[ht]
\centering
\begin{subfigure}{0.40\textwidth}
\begin{minted}{lua}
function counter(initial)
    local count = initial
    return {
        get = function()
            return count
        end,
        increment = function()
            count = count + 1
        end
    }
end

local c = counter(41)
print(c.get()) --> 41
c.increment()
print(c.get()) --> 42
\end{minted}
\caption{Using a \lua{get()} function and a direct \lua{count} upvalue.}
\label{lst:lua_private_a}
\end{subfigure}
\hspace{0.09\textwidth}
\begin{subfigure}{0.40\textwidth}
\begin{minted}{lua}
function counter(initial)
    local self = {}
    self.count = initial
    function self.increment()
        self.count = self.count + 1
    end
    return setmetatable({}, {
        __index = self,
        __newindex = function() end
    })
end

local c = counter(42)
print(c.count) --> 42
c.count = 10
print(c.count) --> 42
getmetatable(c).__index.count = 10
print(c.count) --> 10
\end{minted}
\caption{Using a table with a no-op \lua{__newindex} metamethod. With a detour, the value can still be modified from the outside.}
\label{lst:lua_private_b}
\end{subfigure}
\caption{Two ways of making values private using closures: \lua{count} cannot be accidentally modified from the outside.}
\label{lst:lua_private}
\end{figure}

Both of these methods work for preventing accidentally modifying a task's value. For the proof of concept however, we will not be using any of these options. While that makes it possible to violate a task's autonomicity, that will not happen in normal use.

\subsubsection{Value and stability}
When using functions or coroutines as tasks, we can choose to return or yield the task value and its stability separately since Lua allows returning multiple values. Closer to what iTasks does, we could also return a table containing the value and the stability. Returning the task value and its stability separately is more idiomatic in Lua. However, this can lead to problems where the value and stability need to be passed around. Especially for the \lua{parallel} combinator because its task value is a list of task values.

We will keep the actual value and the stability separate and only pack them together when needed. In the proof of concept, this only happens in the \lua{parallel} combinator. This does make it inconsistent with the rest though.

% \newpage
\section{Type representation}\label{section-type-representation}
Because we decided in section \ref{section-types-matching} that task continuations have an associated type, we need some way to represent Lua types at runtime. This typing information is used by the step combinator's type match function to decide which task continuation it should choose.
Lua has the \lua{type} function that returns the type of the value passed as a string: \lua{type(42) == "number"}. The problem is that this does not give us detailed enough information for tables; \lua{type({10, 20})} and \lua{type({hello = "world"})} both result in just \lua{"table"}.

Tasks and editors require a more elaborate system that can distinguish types of composite values. We need to consider the way these types are written, how they are represented or stored at runtime, and how they are compared against each other. We will elaborate on multiple ways to solve the first two considerations now, how to compare types is left for section \ref{section-combinators-type-matching}.

\subsection{LuaRocks libraries}\label{section-task-types-luarocks}
When looking for Lua libraries, I primarily used LuaRocks\footnote{\label{footnote-luarocks}\url{https://luarocks.org/}}, which is the most used Lua package manager and package repository. There are a number of libraries that come up when searching for ``types''. Three of them have some way to represent composite types at runtime:
luastruct\footnote{\label{footnote-luastruct}\url{https://luarocks.org/modules/UlisseMini/luastruct}},
struct.lua\footnote{\url{https://github.com/mpatraw/struct.lua}} and
Typed\footnote{\label{footnote-typed}\url{https://luarocks.org/modules/SovietKitsune/typed}}.

\subsubsection{luastruct and struct.lua}
luastruct and struct.lua represent types at runtime by a default value. The example from the LuaRocks description\footref{footnote-luastruct} of luastruct describes the type of a table with a \lua{name} field of type \lua{string} (by default \lua{"default name"}) and an \lua{age} field of type \lua{number} (default \lua{0}):

\medskip
\begin{minted}{lua}
local person = struct {
    name = "default name",
    age  = 0
}
\end{minted}

struct.lua works in the same way, and this example is also valid there. This may be a very simple way to store composite types at runtime, but it has the obvious downside that every field must have a default value. For editors, this is not that big of a problem. But for specifying what type a task accepts, this can be very inconvenient. Furthermore, in this place the actual default value does not have any use: only its type will be used. A bigger problem for representing types of tables in this thesis is that these libraries are only about \textit{structs}; they do not have a way to represent arrays.

\subsubsection{Typed}
Typed is a library for checking a function's arguments. It gives formatted error messages containing information on what type was expected. The error messages are not interesting for this thesis, but how it represents composite types is. Arrays can be represented like the string \lua{"number[]"}, maps are written as \lua{"table<string, boolean>"}. When multiple types are valid, they can be written as \lua{"string | number"}. For more complicated types like what LuaStruct and Struct.lua do, it uses schemas, for example a table that contains the string field \lua{name} and a numeric field \lua{id} is written like this: \lua{typed.Schema('test'):field('name', 'string'):field('id', 'number')}.

\subsection{Lua extensions}
% \subsubsection{Typed Lua}
Maidl, Mascarenhas and Ierusalimschy \cite{maidl2014typed} designed a gradually typed extension of Lua called Typed Lua. It does not keep types at runtime, but it does have its own way of representing these types in code.

% \subsubsection{Pallene}
Pallene, developed by Gualandi and Ierusalimschy \cite{gualandi2020pallene}, is a typed subset of Lua. In contrast to Typed Lua, it does sometimes keep types for runtime type checks.

% \subsubsection{Teal}
Teal\footnote{\url{https://github.com/teal-language/tl}} is a language that compiles to Lua, implemented in Lua. It has an online playground\footnote{\url{https://teal-playground.netlify.app}} that shows that types are removed at runtime.

% \subsubsection{Luau}
Unlike Teal, Luau\footnote{\url{https://luau-lang.org/}} does not compile to Lua but has its own interpreter. Like Luau however, it also does not keep types at runtime.

\subsection{Other languages}
\subsubsection{TypeScript}
TypeScript\footnote{\url{https://www.typescriptlang.org/}} is a language that transpiles to JavaScript. Like Typed Lua, Teal and Luau, its types get removed at compile time. We can still learn from the way types are written, though.

\subsection{Typed library}
The Typed library library is the most complete of the three libraries, so we will use it in the proof-of-concept for representing types at runtime. The matching of types will initially also be done by the library, but later on we will design a custom match algorithm. While the library is more complete than the rest, it is still missing some non-essential features we would like to have such as being able to describe a table which both has predetermined fields and is also an array. Implementing these is out of scope for this thesis, but the design decisions themselves will be considered in section \ref{section-combinators-type-matching}.

% \newpage
\section{Task combinators}\label{section-combinators}

\subsection{Combinators and operators}
Combinators are common in functional languages like Clean, where it is possible to define custom operators for them. For instance, iTasks defines an infix operator \clean{>>*} for the \clean{step} function. In order to be able to easily compare the proof of concept to iTasks, we want to come close to the notation as used in iTasks.

Lua does not allow defining custom operators, but you can change the behaviour of the pre-existing operators. To do this, we define a \lua{task} table and use it not only to define all combinators, but also as a metatable for tasks. For changing the behaviour of, for example, the \lua{&} function, we define the \lua{__band} metamethod in this table. We let all tasks inherit from this prototype table using the \lua{__index} metatable entry, see listing \ref{lst:lua_task_metatable}.

\begin{figure}[ht]
\centering
\begin{minted}{lua}
local task = {}
task.__index = task
task.__band = function() --[[ ... ]] end

local myTask = setmetatable({}, task)
\end{minted}
\caption{A simplified example showing the basic structure for inheriting the prototype and defining custom operator behaviour.}
\label{lst:lua_task_metatable}
\end{figure}

% While custom infix operators make combinators easier to use, they are not a requirement. We can define them as normal Lua functions, like the plain \clean{step} function in iTasks.

\subsection{Parallel}
If we bring the \clean{parallel} signature from iTasks down to its essence, we get listing \ref{lst:clean_parallel}. It takes a list of tasks, the task it returns has as its value a list of values of the original tasks.

\begin{figure}[ht]
\centering
\begin{minted}{clean}
parallel :: [Task a] -> Task [TaskValue a]
\end{minted}
\caption{The simplified \clean{parallel} combinator's signature.}
\label{lst:clean_parallel}
\end{figure}

Each time the \lua{parallel} task is resumed (we decided in section \ref{section-task-values-fn-coroutine} that it is a coroutine), it resumes the input tasks one by one and updates its list of task values. Because the resulting task needs to also contain the task values' stability, the value of the \lua{parallel} task is a list of task value--stability pairs. Listing \ref{lst:clean_parallel_example} shows a very simple example of parallel ``and.''

\begin{figure}[ht]
\centering
\begin{minted}{clean}
(return "A" -&&- return "B") >>- (\x -> viewInformation [] x)
\end{minted}
\caption{A simple example of using \clean{parallel}. This shows ``A'' and ``B'' in the output.}
\label{lst:clean_parallel_example}
\end{figure}

\subsection{Step}
The step combinator executes one task and chooses another task to execute using the observable task value of the first task. The result of the step combinator is a task that has the value of the selected follow-up task. It is called \textit{step} because when it can execute one of the follow-up tasks, it steps to that task and does not go back anymore.

In iTasks, the step combinator expects a list of \textit{task continuations}. Such a continuation defines a task that should be executed when some event happens. Such an event can be when a task has a stable value or when a task has a value that matches some predicate (\clean{OnValue}). It can also be when the user presses some button like `yes', `no', `ok' or `cancel' (\clean{OnAction}). Listing \ref{lst:clean_step_onvalue_onaction} shows an example of using multiple \clean{OnValue} continuations. The step combinator only steps to a continuation if its predicate holds. If we simplify its signature from iTasks, we get listing \ref{lst:clean_step}.

\begin{figure}[ht]
\centering
\begin{minted}{clean}
step :: (Task a) [TaskCont a (Task b)] -> Task b

:: TaskCont a b
    = OnValue ((TaskValue a) -> ? b)
    | OnAction String ((TaskValue a) -> ? b)
\end{minted}
\caption{The simplified \clean{step} combinator's signature, together with the type definition of \clean{TaskCont} (also simplified).}
\label{lst:clean_step}
\end{figure}

Each time the \lua{step} task is resumed before stepping, it resumes the first task and tries to find a matching continuation task. When one such continuation task is found, it steps. Now, the \lua{step} task acts as a proxy to the continuation task: it resumes the continuation task and updates its own task value and stability to match that task.

\begin{figure}[ht]
\begin{subfigure}{\textwidth}
\centering
\begin{minted}{clean}
enterInformation [] >>* [
    OnValue (ifValue isPalindrome (showInput "palindrome: ")),
    OnValue (ifValue isGreeting (showInput "greeting: "))]
\end{minted}
\caption{Using the step combinator with \clean{OnValue} in iTasks. It will automatically step once the user input is either a palindrome or a greeting. \clean{isPalindrome} and \clean{isGreeting} are defined elsewhere, their implementation is not important. (A greeting is something like ``hello'' or ``I am ...'')}
\label{lst:clean_step_onvalue}
\end{subfigure}
\begin{subfigure}{\textwidth}
\centering
\bigskip
\begin{minted}{clean}
enterInformation [] >>* [
    OnAction (Action "Check palindrome")
        (ifValue isPalindrome (showInput "palindrome: ")),
    OnAction (Action "Check greeting")
        (ifValue isGreeting (showInput "greeting: "))]
\end{minted}
\caption{The same example as (a), but with \clean{OnAction}: it will only step when the user clicks ``Check palindrome'' or ``Check greeting.''}
\label{lst:clean_step_onaction}
\end{subfigure}
\caption{\clean{OnValue} and \clean{OnAction} in iTasks. \clean{showInput} is a convenience wrapper around the iTasks function \clean{viewInformation}.}
\label{lst:clean_step_onvalue_onaction}
\end{figure}

\subsection{Type matching}\label{section-combinators-type-matching}
The type of values that a continuation expects will need to be attached to the continuation, in the format just described in section \ref{section-type-representation}. To decide what continuation to step to in Lua, we use a type matching function. As hinted at in section \ref{section-types-matching}, there are many different ways for a type matching function to work. The considerations as well as the choices for this proof of concept and the reasoning behind the choices are outlined here. The syntax used here is hypothetical.

\subsubsection{Best match or first match}
When there are multiple continuations that match the current task value, we need to decide which of the continuations to execute. This possibility of having multiple continuations that match is also present in iTasks, where the first \clean{OnValue} or otherwise the first \clean{OnAction} match is used. Actually, in iTasks all continuations need to accept exactly the same type so it is not possible to let the system automatically find a ``best'' match, only manually. This is easier to do in dynamically typed languages like Lua.

We can define a \textit{better} match to be a more \textit{specific} one: \lua{number} is more specific than \lua{string | number} (a union), because the first one does not accept strings. \lua{table<string, number>} (a table with \lua{string} keys and \lua{number} values) is more specific than just \lua{table}, and a table \lua{{id: number, age: number}} (a struct) is even more specific than both of these.

% https://link.springer.com/content/pdf/10.1007/3-540-52592-0_68.pdf

We can formalise this intuitive relation, let's write $T_1 < T_2$ if $T_2$ is more specific than $T_1$. To be able to use this relation in Lua with the \lua{table.sort} function, it needs to be a strict partial order \cite[\S 6.6]{luareferencemanual}: it must be irreflexive, asymmetric and transitive. If some $T_1$ and $T_2$ do not match any of the following rules, they are either not comparable or equivalent. $T$ denotes any type, $t$ is any type except unions, $F$ and $G$ are pairs of key name and value type, and $k$ is a string key. $T\ |\ T$ (same type on left and right side) is equal to just $T$. Order does not matter for union types: $T_1\ |\ T_2$ is equal to $T_2\ |\ T_1$. A struct with no pairs is equal to a table. Note that relation defined here is intended to be simple, so it does not include things like tuple types or a specified list length.

The \lua{any} type is the least specific because it matches all types:
\[ \mathrm{any} < T \qquad\mathrm{if~} T \neq \mathrm{any} \]

A union of two types is less specific than a single type:
\[ T_1\ |\ T_2 < t_3 \]

For two unions with a corresponding type, one is less specific than the other if the non-corresponding type is less specific:
\[ T_1\ |\ T_2 < T_1\ |\ T_3 \qquad\mathrm{if~} T_2 < T_3 \]
% \[ T_1\ |\ T_2 < T_3\ |\ T_4 \quad\mathrm{if~} \dots \] % T_1 < T_3 \mathrm{~or~} T_2 < T_4 \]

A table of any type is less specific than one with a list type specified:
\[ \mathrm{table} < \mathrm{table}(T) \]

The same for a table that has a key and value type specified:
\[ \mathrm{table} < \mathrm{table}(T_1, T_2) \]

A list is less specific than another list if their element types are less specific:
\[ \mathrm{table}(T_1) < \mathrm{table}(T_2) \qquad\mathrm{if~} T_1 < T_2 \]

A table with string keys and a set value type is less specific than a struct type (given that the struct type is not empty):
\[ \mathrm{table}(\mathrm{string}, T) < \{F_1, \dots, F_n\} \]

For two struct types with a corresponding pair of key and value-type, one is less specific than the other if the rest of the struct types is less specific:
\begin{multline*}
\{F_1, \dots, F_n, k: T\} < \{G_1, \dots, G_m, k: T\} \\
\mathrm{if~} \{F_1, \dots, F_n\} < \{G_1, \dots, G_m\}
\end{multline*}

For two struct types with the same number of pairs and a corresponding key, one is less specific than the other if the value-type is less specific and the rest of the struct is less specific:
\begin{multline*}
\{F_1, \dots, F_n, k: T_1\} < \{G_1, \dots, G_n, k: T_2\} \\
\mathrm{if~} T_1 < T_2 \mathrm{~and~} \{F_1, \dots, F_n\} < \{G_1, \dots, G_n\}
\end{multline*}

% \begin{eqnarray*}
% T < U &=& 0 \\
% T\ |\ U < V &=& 1 \\
% \mathrm{table}(T) < \mathrm{table} &=& 1 \\
% \mathrm{table}(T, U) < \mathrm{table} &=& 1 \\
% \{F_1, \dots, F_n\} < \mathrm{table}(T, U) &=& 1 \\
% \{F_1, \dots, F_n\} < \{F_1, \dots, F_m\} &=& 1 \quad\mathrm{if~} n < m,\ 0 \mathrm{~otherwise} \\
% \end{eqnarray*}

\subsubsection{Matching lists: types and order}
A list in Lua can contain values of differing types at once. What happens if the actual list contains the right types but in a different order than asked for?
% If we would allow changing the order of the types in a list, a continuation function expecting a string and a number could receive a number and a string instead.
This goes wrong if the position of elements in the list has meaning. Typescript calls this tuple types\footnote{\label{footnote-typescript-tuple-types}\url{https://www.typescriptlang.org/docs/handbook/2/objects.html\#tuple-types}}. An example of this is a continuation accepting a date as a table of \lua{{number, string, number}} (year, month, day). When it receives a \lua{{number, number, string}} instead, it can not know which number is the day and which is the year. Therefore, a list with a different order of types should never match.

% The Typed library does not have functionality to define types for specific list indices, only for the entire list. Typescript calls this tuple types\footnote{\label{footnote-typescript-tuple-types}\url{https://www.typescriptlang.org/docs/handbook/2/objects.html\#tuple-types}}.

\subsubsection{Matching lists: length}
If the continuation specifies a list length and if the actual list is longer than this length, does it still match? List elements may have semantics, so if we choose to match a list that is longer than needed, we may discard important information. This can happen for example when we have a 3D vector that is represented as a list of its coordinates. If we have two continuations, one for 2D vectors and one for 3D vectors, we should not choose the 2D vector continuation. To prevent situations like this, we should not match lists that are longer than requested. The best-match algorithm described above does not include list length, so using that does not help.

% Typescript works around the issue in this example by providing tuple types\footref{footnote-typescript-tuple-types}.
% The Typed library does not have tuple types or functionality for specifying list length.

\subsubsection{Matching tables}
Analogous to list length: when a table has more fields than required, does it match? The same 2D/3D vector example applies here, but with tables containing the fields \lua{x}, \lua{y} and \lua{z}.
% If we do not match a table with extra fields, it can happen that we do not choose a continuation that is perfectly well able to handle the given table. If we do match a table that has extra fields when using first-match, we may never get to a more specific continuation.
This problem can be solved in two ways: by using the best-match algorithm described above or by manually ordering the continuations, placing the continuation accepting a table with the fewest number of fields last.

\pagebreak
\section{Editors and user interface}\label{section-editors-ui}
There are many different ways of interfacing with users. iTasks uses a webpage for instance. But there are other graphical interfaces, as well as non-graphical ones. They all differ in usability for the user and ease of programming. We explored a JSON-based interface in section \ref{section-editors-lua}, which would be an especially non-user-friendly user interface.

\subsection{HTML page}
There is one well-known Lua library for and for interacting with the DOM through Javascript: Fengari\footnote{\url{https://fengari.io/}}. Fengari implements a Lua VM, so Lua code runs in the browser. We can also generate HTML using h5tk\footnote{\url{https://luarocks.org/modules/forflo/h5tk}} and serve it using
LuaSocket\footnote{\url{https://luarocks.org/modules/lunarmodules/luasocket}},
http\footnote{\url{https://luarocks.org/modules/daurnimator/http}},
Fullmoon\footnote{\url{https://github.com/pkulchenko/fullmoon}},
Lapis\footnote{\url{https://luarocks.org/modules/leafo/lapis}},
Lor\footnote{\url{https://luarocks.org/modules/sumory/lor}},
Sailor\footnote{\url{https://github.com/sailorproject/sailor}} or
Pegasus\footnote{\url{https://luarocks.org/modules/evandrolg/pegasus}}.

We will not be going this way, because while it may be the most user-friendly option and cross-platform, we estimate that the amount of work exceeds the scope of this proof-of-concept project and other options are usable enough for a proof of concept.

\subsection{Native application}
A native application looks about the same as a HTML page, but the difference is that interaction does not go via Javascript but via an API written in C. Lua native UI libraries:
fltk4lua\footnote{\url{https://luarocks.org/modules/siffiejoe/fltk4lua}},
TekUI\footnote{\url{https://luarocks.org/modules/luarocks/tekui}},
AbsTK\footnote{\url{https://luarocks.org/modules/pedroalvesv/abstk}\label{footnote-abstk}},
libuilua\footnote{\url{https://luarocks.org/modules/daurnimator/libuilua}},
lui\footnote{\url{https://tset.de/lui/index.html}},
lui\footnote{\url{https://github.com/zhaozg/lui}},
wxLua\footnote{\url{https://github.com/pkulchenko/wxlua}}.

A native application has about the same usability as a webpage. Due to the fact that Lua is built to interoperate with C, it is easier to build a native application than a webpage. For this proof of concept, though, we will use a more simple form of user interface.

\subsection{Terminal text-based UI}
The third way of displaying tasks somewhat graphically is by using a terminal emulator. There are a couple libraries for this:
AbsTK\footref{footnote-abstk},
ltui\footnote{\url{https://luarocks.org/modules/waruqi/ltui}},
termfx\footnote{\url{https://luarocks.org/modules/gunnar_z/termfx}},
lua-tui\footnote{\url{https://github.com/daurnimator/lua-tui}}. The first three are more complete UI-building libraries while lua-tui is more of a toolbox. AbsTK is not available for windows but I also have not been able to get it installed in Ubuntu on WSL yet. Termfx uses the no-longer-maintained termbox which needs Python 2, and I have not gotten that to work either.

For this thesis I chose to work with ltui. It is quite hard to start working with it because it has almost no documentation, but it does have the features needed for displaying tasks and editors. The way in which its example applications are structured is that there is one element of each type: one main dialog, one text input dialog, one output dialog, and so on. When one of these elements is needed, any old contents get replaced and it gets shown on screen.

\subsection{Terminal command-line}
Since a command-line application does not have a graphical interface and is closer to the implementation, this is the least involved way of interfacing with the user. The user can only type commands and the application responds. This however does make it the least user-friendly, but for a minimal proof of concept this matters less. Since it only involves text input and output, it requires no libraries. Because it requires only the minimal extra setup and effort, this will be the initial interface of the proof of concept. Some features are too advanced for such a simple interface, they will only be implemented in a text-based UI.

% \newpage
\section{LTasks}\label{section-ltasks}
To continue the naming scheme of iTasks and mTasks, the proof of concept in this thesis is called LTasks. This section goes into the details of the LTasks library. To prove that the proof of concept is indeed complete for TOP, appendix \ref{appendix} contains some examples in both LTasks and iTasks.

\subsection{Task combinators}
While iTasks provides a lot of combinators, we do not need that for a proof of concept so LTasks includes only the essential combinators and some convenience wrappers around them. Here is the list, along with their operators in LTasks or their equivalent in iTasks:
\begin{itemize}
    \item \lua{constant} (\clean{return} in iTasks)
    \item \lua{step} (\lua{~} in LTasks, \clean{>>*} in iTasks), \lua{stepStable} (\clean{>>-} in iTasks) and \lua{stepButtonStable} (\clean{>>?} in iTasks)
    \item \lua{parallel}, \lua{anyTask}, \lua{parallelAnd} (\lua{&} in LTasks, \clean{-&&-} in iTasks), \lua{parallelOr} (\lua{|} in LTasks, \clean{-||-} in iTasks), \lua{parallelLeft} (\clean{-||} in iTasks) and \lua{parallelRight} (\clean{||-} in iTasks)
    \item \lua{transform} and \lua{transformValue} (\clean{@} in iTasks)
\end{itemize}

These are the most important functions for building a TOP system, as we defined in section \ref{section-top-lua}.

\subsubsection{Step}
\dots

\subsubsection{Parallel}
\dots

\subsubsection{Custom operators}
In Clean it is common to define lots of operators. For example, there are eight different operators for variations of the step combinator. Lua does allow for changing the behaviour of the standard operators, but only up to a point. For example, the result of the comparison operators like \lua{<} is always converted to a boolean \cite{luareferencemanual}. Perhaps the most notable library that uses operators with custom behaviour is LPeg\footnote{\url{http://www.inf.puc-rio.br/~roberto/lpeg/}}. It is not so common to redefine the behaviour of the operators in Lua, so LTasks only uses three operators: \lua{~}, \lua{&} and \lua{|}. Originally the operator for \lua{step} was \lua{..}, which more resembles its original meaning (concatenation, putting strings after each other). It was changed because its right-associativity does not play well with chaining multiple operators.

\subsection{Type matching}
Instead of implementing a type matching algorithm defined in section \ref{section-combinators-type-matching}, we just used the Typed\footref{footnote-typed} library to compare types, which is very strict in what it matches. There is a Lua library for matching data structures called Tamale\footnote{\url{https://luarocks.org/modules/luarocks/tamale}}, however it is not made for matching types so it is also strict in what it matches. Therefore we do not use Tamale. \dots

\subsection{Editors}
\dots

\subsection{User Interface with LTUI}
For simplicity with working with LTUI, I decided to only ever have one UI element of a type at once. Instead of creating a new element every time, the old one is re-used, displayed, and hidden when no longer needed. These re-used elements are defined and created once in \lua{ltuiApp.lua}.
The module \lua{ltuiElements.lua} provides functions that use these reusable elements and set the contents like the task name or the current value.
\lua{ltuiEditor.lua} is the module that then converts these editors into tasks so they can be used with TOP. This module provides the same functions with the same parameters as \lua{terminalEditor.lua}, which provides editors that use standard I/O as a command-line interface instead of a textual UI.

\begin{figure}
    \centering
    \includegraphics[width=\textwidth]{img/screenshot-ltui.png}
    \caption{The textual UI showing a table editor in the background with a number input dialog in the foreground.}
    \label{fig:ltask_ui}
\end{figure}

% \newpage
\section{Wrap-up}\label{section-wrapup}
We explored a number of design decisions in this chapter, and decided on them for the LTasks implementation.
We defined that the proof-of-concept needs to have a basic implementation of tasks that can be composed sequentially and in parallel, and it needs to have editors with a UI (start of the chapter).
For handling typed tasks and editors, we decided to implement structural type matching (section \ref{section-types}).
We make use of coroutines by modelling the task as a table containing a coroutine, the task value and its stability (section \ref{section-task-values}).
The types are represented at runtime using the Typed library (section \ref{section-type-representation}).
We formalised a type specificity algorithm in section \ref{section-combinators-type-matching}. We noted what happens when lists or tables have more elements than required, and when lists have types in a different order than required.
For editors, we use a textual UI instead of a HTML page, a native application or a command-line application (section \ref{section-editors-ui}).
In the previous section (\ref{section-ltasks}) we show how the LTasks implementation works.

% \begin{itemize}
%     \item task types: structural type matching
%     \item task values: coroutine in table
%     \item type representation: Typed library
%     \item combinators, type matching
%     \item editor UI: text-based UI
%     \item LTasks implementation
% \end{itemize}

