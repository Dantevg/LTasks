\section{Task Oriented Programming}\label{section-top}

% \cite{tosca2020semantic}

In Task Oriented Programming \cite{plasmeijer2012task}, a task is just like a task in the real world: a description of something that needs to be done, an abstract unit of work. A task can have an observable intermediate value and access to shared information. Some tasks are to be performed by a human and some can be done by a computer. By composing tasks in various ways, it is possible to create complex applications.

There are two implementations of TOP: iTask \cite{plasmeijer2007itasks}, written in the functional language Clean, is used for developing interactive distributed applications. mTask \cite{koopman2018task, lubbers2019multitasking}, written in Clean and C++, is used for IoT devices which are constrained in their resource usage. Both of them are a shallowly-embedded domain-specific language (EDSL). In this thesis we will primarily be comparing against iTasks.

A TOP implementation must provide the concept of tasks, ways to compose the tasks where one task can read the value of another task, shared data sources and for interactive TOP systems also some form of user interface.
% When a task is in progress, its value can already be observed.
% They vary from simply displaying text to running an entire application.
% The underlying technical details are taken care of automatically by the TOP framework.

% \subsection{Comparison to other paradigms}
% In procedural programming, you structure your code using \textit{procedures}. In object oriented programming you use \textit{objects}, and in functional programming you use \textit{functions}. Analogously, with task oriented programming you use \textit{tasks} to structure your code.

% In imperative programming, you tell the computer \textit{how to do} a computation. In functional programming, you tell it \textit{what needs to be done}. In task oriented programming, you tell it \textit{what the result needs to be} or \textit{how information relates to each other}.

% \subsection{TOP implementations}
% At the moment, there are two implementations of TOP: iTasks in the functional language Clean, and mTasks in Clean and C++. The latter one is designed for IoT devices which are constrained in their resource usage. Both of them are a shallowly-embedded domain-specific language (EDSL). In this thesis we will primarily be comparing against iTasks.

% A TOP implementation must provide the concept of tasks, ways to compose the tasks where one task can read the value of another task, shared data sources and some form of user interface.

% \newpage
\subsection{Task value}\label{section-top-task-value}
Tasks can have an \textit{intermediate value}. A task value can be in one of three states: \textbf{no value}, \textbf{unstable} value or \textbf{stable} value \cite[\S 4.3]{achten2013introduction}. A task can switch between no value and an unstable value, and it can switch to a stable value. Tasks with a stable value have a final result and do not change anymore. Figure \ref{fig:task_states} shows this graphically. Even when a task has no stable value yet, its intermediate value can be observed. Tasks can observe the value of other tasks.

In iTasks, tasks can also throw \textit{exceptions}, implemented as returning an exception value \cite[\S 3.1.1]{plasmeijer2012task}. When it is clear that a task will never result in a meaningful value, it can raise an exception. This can happen for instance when a network connection fails.

\begin{figure}
    \centering
    \begin{tikzpicture}[>={Stealth[scale=1.5, inset=0cm]}]
        \node[state] (no_value)       at (0,3) [minimum size=2cm, align=center] {\textit{no}\\value};
        \node[state] (unstable_value) at (4,3) [minimum size=2cm, align=center] {\textit{unstable}\\value};
        \node[state] (stable_value)   at (2,0) [minimum size=2cm, align=center] {\textit{stable}\\value};
        
        \path[->] (no_value)       edge [bend right] node {} (unstable_value)
                                   edge node {} (stable_value)
                  (unstable_value) edge [bend right] node {} (no_value)
                                   edge node {} (stable_value);
    \end{tikzpicture}
    \caption{The possible states of task values}
    \label{fig:task_states}
\end{figure}

\subsection{Task composition}\label{section-top-composition}
Tasks can be \textit{composed} in multiple ways, falling in one of these categories: \textbf{sequential} composition and \textbf{parallel} composition. They both use the fact that a task's value can be observed.
Sequential composition is named \texttt{step} in iTasks. We can provide it with multiple continuation tasks, one of which will be executed based on the observed task value.
Parallel composition executes all provided tasks. The values of these tasks are combined to form the value of the parallel task, and these values can also be observed by the provided tasks.
There is a way to create tasks with a stable value, called \texttt{return} in iTasks. The value and type of a task can be transformed with iTasks' \texttt{@} operator.

% With sequential composition (named \texttt{step} in iTasks), first task 1 is started before starting task 2 with the value of task 1. In parallel composition, both tasks 1 and 2 need to be done, or either task 1 or 2.

We use the example of having breakfast, adapted from Naus \cite{naus2020assisting}. To make breakfast, you first make something to drink. This can be either tea or coffee. You also make something to eat, a sandwich in this example. You can do that already while you are waiting for your drink to be ready. When you have your drink and your sandwich, you can eat it. The whole operation of having breakfast can be seen and modelled as a task, composed of smaller tasks.

The breakfast example using the combinators \mintinline{clean}{>>-} (sequential), \mintinline{clean}{-&&-} (parallel and) and \mintinline{clean}{-||-} (parallel or) from iTasks looks like listing \ref{lst:clean_breakfast}.

\begin{figure}[ht]
\centering
\begin{minted}{clean}
(makeTea -||- makeCoffee) -&&- makeSandwich >>- eatBreakfast
\end{minted}
\caption{Example: making breakfast.}
\label{lst:clean_breakfast}
\end{figure}

% Or using a function:
% \begin{minted}{clean}
% makeBreakfast :: Task Drink -> Task Food -> Task (Drink, Food)
% makeBreakfast makeDrink makeFood = makeDrink -&&- makeFood

% makeBreakfast (makeTea -||- makeCoffee) makeVegetableScramble
%     >>- enjoyMorning
% \end{minted}

A task composed sequentially does not have to wait for the first task to complete before starting. It can even be the case that task 2 becomes stable when task 1 still has an unstable value. For example, task 1 could give an overview of free hospital beds. Task 2 could then decide to continue (have a stable value) once there is a suitable bed available, while the overview of hospital beds will never be stable. This situation looks like this in iTasks:

\bigskip\noindent
\begin{minted}{clean}
hospitalBeds >>* [OnValue (ifValue (\beds = length beds > 0)
    (\beds = return (hd beds)))]
\end{minted}

\bigskip
Or think of a task that waits for a specific time before returning a value. Task 1 (a task that yields the current time) will always have an unstable value (because time keeps changing), but eventually the waiting task becomes stable.

\subsection{Shared data sources}\label{section-top-sds}
Tasks are \textit{distributed} and \textit{concurrent}. For instance, in a multi-user system, two tasks that are composed in parallel could be done by two different users at the same time. Note that a TOP implementation does not have to be a multi-user system. While iTasks is multi-user, mTasks is not (because having users makes little sense in an IoT environment).

Sharing data between tasks is done with shared data sources like files or sensors. These can interact with tasks. Not all shared data sources have to be both readable and writable; the current date and time are examples of read-only shared data sources.
When tasks are composed in parallel, they get a read-only shared data source that reflects the current value of each other \cite[\S 2.1]{plasmeijer2012task}. The following is an example of a SDS that is both readable and writable, letting the user input a list of words and immediately showing the sentence made from these words.

\bigskip\noindent
\begin{minted}{clean}
wordsSDS :: SimpleSDSLens [String]
wordsSDS = sharedStore "wordsSDS" []

wordsTask = (updateSharedInformation [] wordsSDS <<@ Title "enter words")
    -|| (viewSharedInformation [ViewAs (foldl (+++) "")] wordsSDS
        <<@ Title "sentence view")
\end{minted}

\subsection{User interface}
Interaction of tasks with humans happens with interactive tasks called \textit{editors}. An editor task is a bridge between the internal world of tasks and the external real world.
A TOP framework automatically generates an appropriate user interface with these editors. They also allow users to interact with shared data sources.
% How the user interface is shown is independent from the task structure: a change in the tasks does not require you to change the layout, and a change in layout does not need the underlying task structure to change.

An editor task never has a stable value. When an editor task is composed sequentially, even when the user has pressed a ``continue'' button and the sequential task has moved on to the next task, the editor stays unstable behind the scenes.

User interfaces of combined tasks are composed of the user interfaces of the components. For example in iTasks, if two tasks assigned to the same user are combined in parallel, they are shown next to each other. \cite[\S 4.2.4]{naus2020assisting}

Since iTasks is written in the statically typed language Clean, the possible values a task can have are predetermined by its type. This allows us to make use of the existing HTML form input fields when generating a web interface. This is useful for two reasons. First, these form fields only allow valid input, i.e. you can't input arbitrary text in a number field. Second, they improve usability by adapting the input method, for instance displaying a date picker.

This generation of different HTML form fields is done automatically in iTasks. This is possible because Clean has generic types and is statically typed---types are available statically, before any value is present.

\subsection{Implementation in iTasks}
Since iTasks is written in the functional language Clean, tasks are modelled as functions. Functions in Clean are pure; they cannot mutate existing values. Instead, a task is implemented as a function that can process an event, using the information stored in the \clean{IWorld} environment, and that results in a new version of itself, the task value, any user interface changes, and the \clean{IWorld}.
In listing \ref{lst:clean_task_taskresult_types} we can see that the task function (\clean{Task}) returns a \clean{TaskResult} and an \clean{IWorld}. \clean{IWorld} carries information about the entire TOP application such as the current clock, and we will not go into it further.

A task value is either \clean{NoValue}, or \clean{Value} together with a value and a boolean \clean{Stability} signifying whether it is stable. Note that this models the three possible states from \ref{section-top-task-value} correctly: a task with no value also has no stability.

\begin{figure}[ht]
    \centering
    \begin{minted}{clean}
    :: Task a (=: Task` (Event -> TaskEvalOpts -> *IWorld
            -> *(TaskResult a, *IWorld)))
    
    :: TaskResult a
        = ValueResult !(TaskValue a) !TaskEvalInfo !UIChange !(Task a)
        | ExceptionResult !TaskException
        | DestroyedResult
    
    :: TaskValue a
        = NoValue
        | Value !a !Stability
    \end{minted}
    \caption{The types \clean{Task}, \clean{TaskResult} and \clean{TaskValue} in Clean.}
    \label{lst:clean_task_taskresult_types}
\end{figure}
