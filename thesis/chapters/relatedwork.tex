\chapter{Related Work}\label{relatedwork}
% In this chapter you demonstrate that you are sufficiently aware of the
% state-of-art knowledge of the problem domain that you have investigated as
% well as demonstrating that you have found a \emph{new} solution / approach / method.

There are currently two implementations of TOP.
The iTask system \cite{plasmeijer2007itasks} is a TOP implementation for creating distributed multi-user systems.
The mTask system \cite{koopman2018task, lubbers2019multitasking} is not an interactive one like iTasks, but it is meant for IoT devices, which are constrained in their resource usage.

Naus describes TopHat \cite{naus2020assisting}, a TOP language with formal semantics. They include a clear description of the core TOP features.

There has been some research on adding a type system to Lua. Gualandi and Ierusalimschy introduce a typed companion language to Lua called Pallene \cite{gualandi2020pallene}, and Maidl, Mascarenhas and Ierusalimschy have developed Typed Lua \cite{maidl2014typed}, which is an optionally typed language.
